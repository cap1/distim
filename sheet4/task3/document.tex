\documentclass{article}
\usepackage[utf8]{inputenc}
\usepackage{graphicx}
\usepackage{amsmath}
\usepackage{booktabs}
\usepackage{textcomp}
\usepackage{multirow}
\usepackage{color}

%use (a) for numbering subsections
\renewcommand*\thesubsection{(\alph{subsection})}
\title{Aufgabenblatt 4}
\author{Christian Müller, Ralph Krimmel \& Sebastian Albert }

\begin{document}

\maketitle

\section*{Assignment 3 - Backup and Archive}

\subsection{Propose a backup and recovery solution for the described company!}
	A more centralized approach might be a better solution for the company,
	with a dedicated backup server.
	A mixed backup topology should be implemented as it also reduces administrative overhead
	and reduces the recovery times with fast FC access to the backups.
	The backend should be changed to a virtual tape libary,
	as it offers faster acces times in the case data needs to be recovered.
	To ensure availabilty of e-mail and database services
	the backups should be performed as online backups.

\subsection{Discuss the security conerns in backup environments?}
		The backuped data should be as protected as the live data.
		This means that data should never be send unencrypted over any wire,
		especially when performing a backup
		as a complete ``working copy'' is sent.
		The backuped data should also be stored in a safe way.
		Security can be enhanced by encrypting the data.
		If the data shall be archived for a long time,
		one must keep in mind that encryptions might weaken over time.

		It is also important to restrict physical access to the backup devices,
		as sensitive data could be stolen including the backup media.

\subsection{List and explain the considerations in tape as backup technology!
				What are the challenges in this environment?}
	Tape are a low-cost solution for backups.
	The low price is the main consideration for choosing tape as a backup solution.
	Tapes have been the cheapest waz to store data for a long time
	and therefore a lot of data has been stored in tape libaries already.
	Adding additiononal tape capacity to a storage solution is usually easier 
	then switching the storage technology to disks.
	
	A major challenge with tapes is that they degrade when they are used.
	Magnetic tapes suffer from tear and wear when beeing used.
	When data does not stream apropriately,
	the drive has to repeat back and forth over the tape
	which causes additional degradation of the tape.
	This is known as the shoe-shining-effect.
	Tape libaries have a long response time,
	as a magnetic arm has to grab the tape an insert it into a drive,
	then the tape has to be forwarded to the desired reading position,
	before the first bits of data can be read.
	This long \textsl{load to ready time} is also a challenge when using tapes as backup media.

\subsection{Describe the benefits of using ``virtual tape libaries'' over ``physical tapes''!}
	Virtual tape libaries (VTL) have some key advantages over pyhsical tapes.
	These boil down to the fact,
	that disks are used in stead of tapes
	and thereby eliminating the problems related to using tapes.
	Tapes suffer from tear and wear and the so called shoe-shine-effect,
	which reduces their reliabilty.
	These both features allow faster backup and recovery
	then pyhsical tapes.
	Additionally VTLs offer support for data replication over IP networks,
	so that inexpensive offsite replication is available.
	As VTLs come preconfigured,
	they are easier to integrate than backup-to-disk services,
	especially if a VTL substitutes physical tapes.

\subsection{Describe the data deduplication methods and distinguish source-based and target-base data deduplication!}
	Deduplication can be done on two levels , File-level and subfile-leve,
	and it can be performed at the source or at the target.

	\emph{File-level} deduplication (also: \textsl{single-instance storage}) 
	checks if the file already exits on the storage diveces.
	This is usually done by a cryptography hash-function.
	Subsequent copies of the file are replaced with a pointer to the original file.

	\emph{Subfile level} deduplication,
	takes smaller chunks of the file
	and uses special algorithms to find redundant chunks accross and within the file.
	The chunks can either have a fixed or a variable length.
	This method of deduplication can reduce the amount of stored data,
	as similar files with only slight changes can be stored more effictiantly,
	then with file-level deduplication.

	\emph{Source-based} deduplication eliminates redundant data at the backup-client.
	This method reduces the data send over the network to the backup-device.
	It also provides a shorter bachup window,
	but enhances the load on the backup-client.

	\emph{Target-based} deduplication sends all the data to the bachup-device.
	All calculations take place there.
	This method requires a larger amount of data to be transfered from
	the backup-client to the backup-device.

\end{document}
