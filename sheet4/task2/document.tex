\documentclass{article}
\usepackage[utf8]{inputenc}
\usepackage{graphicx}
\usepackage{amsmath}
\usepackage{booktabs}
\usepackage{textcomp}
\usepackage{multirow}
\usepackage{color}

%use (a) for numbering subsections
\renewcommand*\thesubsection{(\alph{subsection})}
\renewcommand*\thesubsubsection{(\roman{subsubsection})}
\title{Aufgabenblatt 4}
\author{Christian Müller, Ralph Krimmel \& Sebastian Albert }

\begin{document}

\maketitle

\section*{Assignment 2 - Introduction to Business Continuity}

\subsection{}
\begin{eqnarray*} 
	MTBF &=& \frac{\text{Total uptime}}{\text{Number of failures}}  \\
	\text{Total uptime} &=& 1000 - (1000*0.02) \\
	\text{Number of failures} &=& 1 \\
	MTBF &=& 980 \\ 
\end{eqnarray*}

\subsection{}
\begin{eqnarray*}
	IA &=& MTBF/(MTBF-MTTR)\\
	MTFB &=& 35h-0.75h-0.25h \\
	&=& 34h \\
	MTTR &=& 1h \\
	IA &=& 34h/34h+1h \\
	&=& 0.971  \\
\end{eqnarray*}

\subsection{}
A cluster minimizes the RTO because in the cluster the application  and  the data is available on all servers all the time. . In case of a disaster or an outage event this allows to have a really fast failover. There is no need to copy any data to a new location. Due to monitoring and coordination of a cluster, a failover node can nearly instanly take over the application. 
%Implementations: DRBD + heartbeat, AVS (Sun/Oracle)

\subsection{}
Hot site: Has lower RTO, maybe also RPO because it could be synchronized 
Cold site: Higher RTO, site hast to start running, data copied.. 


\subsection{}
\subsubsection{Analysis}
	Consists of three phases:

	\paragraph{Business Impact Analysis}
	\begin{itemize}
	\item Differentiate between critical (= damage is unacceptable) and non-critical organization functions
	\item Critical values are assigned: \textbf{R}ecovery \textbf{P}oint \textbf{O}bjective (the acceptable latency of data that will not be recovered, acceptable amount of data that can be lost is not exceeded) and \textbf{R}ecovery \textbf{T}ime \textbf{O}bjective (the acceptable amount of time to restore the function)
	$\Rightarrow$ Recovery requirements 
	\end{itemize}

	\paragraph{Threat and Risk Analysis}
	\begin{itemize}
	
		\item Potential risks and threats should be documented
		\item Examples: 
			\begin{itemize}
				\item Earthquake, 
				\item Fire, 
				\item Sabotage, 
				\item Cyber attack
				\item ...
			\end{itemize}
		\end{itemize}

	\paragraph{Recovery Requirement Documentation}
	Documenet what is needed for recovery
	Examples: 
	\begin{itemize}
		\item Numbers and types of desks, 
		\item Employees, 
		\item The applications and application data
		...
	\end{itemize}

	\subsubsection{Solution design}
	Identify most effective disaster recovery solution
	\begin{itemize}
		\item minimum application data requirements
		\item minimum time frame in which application data must be available (RTO)
	\end{itemize}

	\subsubsection{Implementation}
		$\Rightarrow$  Execution of the design elements identified in the solution design phase

		\subsubsection{Testing and organizational acceptance}
		
		\begin{itemize}
			\item Usually Biannual or annual schedule → 
			\item Minimizes flaws in recovery requirement, solution design or solution implementation
			\item Includes:
				\begin{itemize}
					\item Crisis command team call-out testing
					\item Application test 
					\item Business process test 
				\end{itemize}
		\end{itemize}


		\subsubsection{Maintenance}
		Three periodic activities
		\begin{itemize}
			\item Information update and testing
				\begin{itemize}
					\item Staffing changes
					\item Departmental changes
					\item ...
				\end{itemize}
			\item Testing and verification of technical solutions
			\item Testing and verification of organization recovery procedures: Changes in systems?, documented work checklists still accurate? 
		\end{itemize}

\end{document}
