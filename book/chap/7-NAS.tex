\section{Network-Attached Storage} % (fold)
\label{sec:network_attached_storage}

\subsection{How can the performance of a NAS be affected by the TCP windows size?} % (fold)
\label{sub:how_can_the_performance_of_a_nas_be_affected_by_the_tcp_windows_size}
	The \emph{TCP window size} is the maximum amount of data
	that can be sent at any time over a TCP connection.
	After the defined amount of data has been sent,
	the sender has to wait for the receivers acknowledgment
	to send more data.
	The receiver acknowledges the actual received amount of bytes,
	so that the sender can send the difference to the window sizes.
	This is done to ensure that all packages have arrived at the receiver.

	If the value of the \emph{TCP window size} is too low,
	the overall throughput of the network might be lower than possible.
	Theoretically the \emph{TCP window size} should be set to:
	\begin{equation}
		available\ bandwidth * round\ trip\ time = TCP\ window\ size
	\end{equation}

% subsection how_can_the_performance_of_a_nas_be_affected_by_the_tcp_windows_size (end)
\subsection{How do jumbo frames affect the NAS performance?} % (fold)
\label{sub:how_do_jumbo_frames_affect_the_nas_perfomance}
	Standard Ethernet frames have a Maximum Transmission Unit (MTU) of 1500 bytes,
	while the larger Jumbo frames can have a MTU of 9000 bytes.
	The MTU describes the maximum size a package can have
	to be sent over an network without fragmentation.

	Larger frames are more efficient,
	as lesser raw bandwidth is used
	and they can hold more payload per package.
	This helps smoothen sudden I/O bursts.\\
	However not all vendors use the same MTU
	and all servers from source to target have to support the MTU/JUmbo frames.
% subsection how_do_jumbo_frames_affect_the_nas_perfomance (end)
\subsection{Research the file access and sharing features of pNFS!} % (fold)
\label{sub:research_the_file_access_and_sharing_features_of_pnfs}
	\emph{pNFS} separates the file meta-data and the data itself.
	A dedicated pNFS server manages the meta-data of the files.
	This includes name, location access control list (ACL) and other attributes.
	The pNFS server enables the client then to directly access
	the data on the storage array.
% subsection research_the_file_access_and_sharing_features_of_pnfs (end)
\subsection{How does file-level virtualization ensure non disruptive file mobility?} % (fold)
\label{sub:how_does_file_leve_virtualization_ensure_nondisruptive_file_mobility}
	With file-level virtualization,
	the clients do not access the files directly on the storage array
	but on the virtualization layer.
	As the virtualization application now manages the location off the files
	on the storage arrays,
	they can be moved in a transparent manner for the clients.
	The file locations are now logical instead of physical.
% subsection how_does_file_leve_virtualization_ensure_nondisruptive_file_mobility (end)

% section network_attached_storage (end)