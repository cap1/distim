\section{Backup and Archive} % (fold)
\label{sec:backup_and_archive}

\subsection{Describe different Backup-types and -granularities.} % (fold)
\label{sub:describe_different_types_of_backup}
	\subsection{Full Backup} % (fold)
	\label{sub:fullbackup}
		Copies the complete data on the production volumes.
		Provides the fastest recovery,
		but takes the largest amount of data
		and the longest time to perform the backup.
	% subsection fullbackup (end)

	\subsection{Incremental Backup} % (fold)
	\label{sub:incremental_backup}
		Incremental backups just contain the data that has accutally changed
		since the last full or incremental backup.
		Takes less space,
		but requires more time to restore
		as it is more complex to perform.
	% subsection incremental_backup (end)

	\subsection{Comulative Backup} % (fold)
	\label{sub:comulative_backup}
		Copies all the data that has changed since the last full backup.
		This method takes longer than an incremental backup
		but it is faster to restore.
	% subsection comulative_backup (end)
% subsection describe_different_types_of_backup (end)

\subsection{List and explain the considerations and challenges in using tape as backup technology!} % (fold)
\label{sub:list_and_explain_the_considerations_and_challenges_in_using_tape_as_backup_technology}
	What are the challenges in this environment?}
	Tape are a low-cost solution for backups.
	The low price is the main consideration for choosing tape as a backup solution.
	Tapes have been the cheapest waz to store data for a long time
	and therefore a lot of data has been stored in tape libaries already.
	Adding additiononal tape capacity to a storage solution is usually easier 
	then switching the storage technology to disks.
	
	A major challenge with tapes is that they degrade when they are used.
	Magnetic tapes suffer from tear and wear when beeing used.
	When data does not stream apropriately,
	the drive has to repeat back and forth over the tape
	which causes additional degradation of the tape.
	This is known as the shoe-shining-effect.
	Tape libaries have a long response time,
	as a magnetic arm has to grab the tape an insert it into a drive,
	then the tape has to be forwarded to the desired reading position,
	before the first bits of data can be read.
	This long \textsl{load to ready time} is also a challenge when using tapes as backup media.
% subsection list_and_explain_the_considerations_and_challenges_in_using_tape_as_backup_technology (end)

\subsection{Describe benefits of using avirtual tape libary over physical tape libary!} % (fold)
\label{sub:describe_benefits_of_using_avirtual_tape_libary_over_physical_tape_libary}
	Virtual tape libaries (VTL) have some key advantages over pyhsical tapes.
	These boil down to the fact,
	that disks are used in stead of tapes
	and thereby eliminating the problems related to using tapes.
	Tapes suffer from tear and wear and the so called shoe-shine-effect,
	which reduces their reliabilty.
	These both features allow faster backup and recovery
	then pyhsical tapes.
	Additionally VTLs offer support for data replication over IP networks,
	so that inexpensive offsite replication is available.
	As VTLs come preconfigured,
	they are easier to integrate than backup-to-disk services,
	especially if a VTL substitutes physical tapes.
% subsection describe_benefits_of_using_avirtual_tape_libary_over_physical_tape_libary (end)

\subsection{What are the benefits and challenges of using cloud storage for archiving?} % (fold)
\label{sub:what_are_the_benefits_and_challenges_of_using_cloud_storage_for_archiving}

% subsection what_are_the_benefits_and_challenges_of_using_cloud_storage_for_archiving (end)

% section backup_and_archive (end)