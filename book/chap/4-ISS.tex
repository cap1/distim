\section{Intelligent Storage Systems} % (fold)
\label{sec:intelligent_storage_systems}

\subsection{Research Cache Cohereny mechanisms and how they address multiple shared cache environments!} % (fold)
\label{sub:research_cache_cohereny_mechanisms_and_how_they_address_multiple_shared_cache_environments}
	In multiple cache environments one either needs a bus that allows inter cache communication (Bus sniffing/snooping)
	or a centralized entity that know about the contents of each cache.
	Each cache page needs a flag to determine if the data is valid
	and has been written to the disk.
% subsection research_cache_cohereny_mechanisms_and_how_they_address_multiple_shared_cache_environments (end)

\subsection{Which type of application benefits most from bypassing the write cache?} % (fold)
\label{sub:which_type_of_application_benefits_most_from_bypassing_the_write_cache_}
	Any application that does write large amounts of sequential data.
	This kind of application does not benefit from a shorter write time,
	which can be produced with a write cache.
% subsection which_type_of_application_benefits_most_from_bypassing_the_write_cache_ (end)

\subsection{Research cache parameters!} % (fold)
\label{sub:research_cache_parameters_}
	\begin{description}
		\item[Cache page size] \hfill \\
			Size of a page in the cache.
			Should be equal to the block in the underlying disk.
		\item[Cache allocation] \hfill \\
			Read or write orientation of the cache.
			Depends on purpose.
		\item[Pre-fetch type] \hfill \\
			What should be cached without request.
			Depends on the application.
		\item[Write aside cache] \hfill \\
			Threshold for the amout of requested I/O operations.
			If it is exceeded the writes go directly to the disk.
	\end{description}
% subsection research_cache_parameters_ (end)

% section intelligent_storage_systems (end)