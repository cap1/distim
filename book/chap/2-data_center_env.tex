\section{Storage Systems}

\subsection*{Why is RAID 1 not a substitute for a backup?} % (fold)
\label{sub:why_is_raid_1_not_a_substitute_for_a_backup_}
	Stores only the current state of the system.
	Backups store distinct points in time to whom the system can be reverted.
% subsection why_is_raid_1_not_a_substitute_for_a_backup_ (end)

\subsection{Research RAID 6 and its second parity computation.} % (fold)
\label{sub:research_raid_6_and_its_second_parity_computation_}
	The second parity can be calculated in various ways.
	The easiest way is to create a second XOR.
	Ususally its is done via these formulas with a shifted Q
	and P being the standard XOR over the stripes.

	\begin{eqnarray}
		P &=& D_0 \oplus D_1 \oplus D_2 \oplus \dotsc \oplus D_n \\
		Q &=& g^0D_0 \oplus g^1D_1 \oplus g^2D_2 \oplus \dotsc g^nD_n 
	\end{eqnarray}
	Where g is a generator of a Galois Field. 
% subsubsection research_raid_6_and_its_second_parity_computation_ (end)

\subsection{Explain the process of data recovery in case of a drive failure in RAID 5.} % (fold)
\label{sub:explain_the_process_of_data_recovery_in_case_of_a_drive_failure_in_raid_5_}
	RAID 5 uses striping of data and distributes the parity information over all disks.
	In the case of a failure of a single  data can be restored by reversing the XOR operation
	that was used to generate the parity information.
	As the partiy information is distributed over all drives in RAID 5,
	some of it is lost if a disk fails.
	This is not a problem becaus it can be generated again form the data itself.
	The recovery works blockwise,
	like the data is stripe accross the drives.
	The remaining data and if availabel the parity information get XORed with each other
	and thereby generates the missing block.
	As described before,
	if the parity information is missing it is can be calculated by XORing the data blocks 
	with themselfs.

	In an ideal case,
	this data is then saved to an hot spare disk,
	which repalacs the disk that failed.
% subsection explain_the_process_of_data_recovery_in_case_of_a_drive_failure_in_raid_5_ (end)

\subsubsection{What are the benefits of using RAID 3 in a backup application?} % (fold)
\label{ssub:what_are_the_benefits_of_using_raid_3_in_a_backup_application_}
	With RAID 3 one benefits from havin fast sequential write and reads like in RAID 0.
	RAID 3 adds parity information to the setup of RAID 1 on a seperate disk.
	This allows to recover from the failure of a one disk in the setup.
	Also RAID 3 does not need huge amounts of additional diskspace,
	as the parity information is the only data generated additionaly.
% subsubsection what_are_the_benefits_of_using_raid_3_in_a_backup_application_ (end)

\subsection{Discuss the impact of random and sequential I/Os in different RAID configurations.} % (fold)
\label{sub:discuss_the_impact_of_random_and_sequential_i_os_in_different_raid_configurations_}

	\subsubsection{RAID 3} % (fold)
	\label{ssub:raid_3}
		\emph{Sequential I/O} can be handle very well by RAID 3,
		as it can spread the I/O with striping over all the disks.
		\emph{Random I/O} reads can be handle very well too,
		but it is not that good with sequential writes.
		This is due to the write penalty,
		which occurs in the creation of the parity information.
		Overall the speed can be described by $(n-1)X$,
		where n is the number of used drives and X the speed of the drives.
	% subsubsection raid_3 (end)

	\subsubsection{RAID 5} % (fold)
	\label{ssub:raid_5}
		\emph{Sequential I/O} can be handle very well by RAID 5,
		as it can be spread over all the disk by striping.
		\emph{Random I/O} can be handle well to.
		The write penalty is the same as in RAID 3,
		but as the parity information are not stored on a dedicated disk,
		it can also be spread.
		The formula described in RAID 3 also holds for RAID 5.
	% subsubsection raid_5 (end)

	\subsubsection{RADI 1/0} % (fold)
	\label{ssub:radi_1_0}
		RAID 1/0 delivers the best performance for \emph{Sequential I/O} and 
		\emph{Random I/O}, as striping can be used without the write penalty
		for the parity information.
		The perfomance increases linear with the number of disks.	
	% subsubsection radi_1_0 (end)
% subsection discuss_the_impact_of_random_and_sequential_i_os_in_different_raid_configurations_ (end)