\section{Introduction to Information Storage} % (fold)
\label{sec:introduction_to_information_storage}

\subsection*{Differentiate structured and unstructured data!} % (fold)
\label{sub:differentiate_structured_and_unstructured_data}
	The content of data bases is structured information,
	as a computer can deduce information from the data.
	Single datafields have relations with each other
	and they are described in a machine readable manner.

	Unstructured is basically everything else,
	like images, pdf files, or html pages.
	Images can be (easily) interpreted by a machine,
	html, xml and excel sheet can be semi-structured.
% subsection differentiate_structured_and_unstructured_data (end)

\subsection{Discuss the benefit of information-centric over server-centric storage architecture!} % (fold)
\label{sub:discuss_the_benefit_of_information_centric_over_server_centric_storage_architecture}
	With an information-centric structure,
	all the data can be shared.
	This means that servers could access all available data,
	in contrast to server-centric architecture,
	where it could only access from the server hosting the information.

	Although it requires a higer bandwith to transport the data to and from the servers,
	several benefits exist.
	These are a central access point to the data for backups.
	The storage capacity can be easier managed, enlarged and utilized.
% subsection discuss_the_benefit_of_information_centric_over_server_centric_storage_architecture (end)

\subsection{What are the attributes of "'Big Data'"?} % (fold)
\label{sub:what_are_the_attributes_of_big_data}
	The term "Big data" describes a certain type and amout of data,
	which is not supported by current software.
	It describes data where sophisticated methods have to be applied,
	to generate and extrac the inheren information.

	Good examples are the data generated by the LHC or genomic data.
% subsection what_are_the_attributes_of_big_data (end)

% section introduction_to_information_storage (end)