\section{IP SAN and FCoE} % (fold)
\label{sec:ip_san_and_fcoe}

\subsection{Describe the process of authentication in iSCSI.} % (fold)
\label{sub:describe_the_process_of_authentication_in_iscsi}
	iSCSI usually handles the authentication with Challenge-Handshake Authentication Protocol (CHAP).
	This protocol requires a shared secret.
	The initiator sends the random value to the target 
	and both calculate the digest of the shared secret and this random value.
	The target sends the digest back to the initiator who completes authentication,
	if the digest and therby the shared secrets are identical.
% subsection describe_the_process_of_authentication_in_iscsi (end)

\subsection{Why should an MTU value of at least 2500 bytes be configured in a bridged iSCSI environment?} % (fold)
\label{sub:why_should_an_mtu_value_of_at_least_2500_bytes_be_configured_in_a_bridged_iscsi_environment}

% subsection why_should_an_mtu_value_of_at_least_2500_bytes_be_configured_in_a_bridged_iscsi_environment (end)

\subsection{Why does the lossy nature of standart Ethernet make it unsuitable for a layered FCoE implementation? How does CEE address this problem?} % (fold)
\label{sub:why_does_the_lossy_nature_of_standart_ethernet_make_it_unsuitable_for_a_layered_fcoe_implementation_how_does_cee_address_this_problem}   	
	Standard Ethernet is lossy,
	which allows frames to be dropped or lost in transmission.
	Usually higher level protocols like TCP take care of the correct order of transmission,
	but they are not present in FCoE and would lead to a large overhead.

	The following technologies are part of CEE to prevent packet loss in FCoE environments:\\
	\emph{Priority-based Flow Control} (PFC) makes use of the IEEE 802.3X Ethernet PAUSE control frame.
	This frame is send by the receiver if his buffers are full,
	to notify the sender and make him stop sending packets
	so that no packet is lost
	As the PAUSE control frame blocks the communication on the entire link,
	PFC creates eight seperate virtual links to prevent a single full buffer from blocking
	all the communication.

	\emph{Congestion Notification} allows the receivin FCoE switches to tell the sender to
	reduce the data transfer rate.
	These notifications are route backwars to the data stream so that the host can limit the rate.
% subsection why_does_the_lossy_nature_of_standart_ethernet_make_it_unsuitable_for_a_layered_fcoe_implementation_how_does_cee_address_this_problem (end)

\subsection{Compare various data center protocols that use Ethernet as the physical medium for transporting storage traffic!} % (fold)
\label{sub:compare_various_data_center_protocols_that_use_ethernet_as_the_physical_medium_for_transporting_storage_traffic}

% subsection compare_various_data_center_protocols_that_use_ethernet_as_the_physical_medium_for_transporting_storage_traffic (end)

% section ip_san_and_fcoe (end)