\section{Local Replication} % (fold)
\label{sec:local_replication}

\subsection{Research various techniques used to ensure consistency of a local replica.} % (fold)
\label{sub:research_various_techniques_used_to_ensure_consistency_of_a_local_replica}
	The easiest way to ensure consistency when performing a replication
	is to hold all I/O to the data.
	This is not always possible as,
	request might time out.
	However one can be sure that at the end of the replication,
	the data is correctly replicated.
	If this technique is applicable depends on the time it
	takes to perform the replication.

	As I/O depends usually on a previous set of I/Os
	it is also possible to capture all I/Os that occurred during the replication.
	These I/Os are then also performed on the replica.
	This ensures that data is still accessible and writable,
	but the replica equals the data at the point in time
	when the replica is finished.
	One has to make sure to perform all dependent I/Os
	and apply them in the correct order.
% subsection research_various_techniques_used_to_ensure_consistency_of_a_local_replica (end)

\subsection{Describe local replication technologies!} % (fold)
\label{sub:describe_local_replication_technologies}

	\subsubsection{Host-based replication} % (fold)
	\label{ssub:host_based_replication}
		\begin{description}
			\item[LVM-based replication] \hfill \\
				The logical volume managers takes care for creating host-level logical volumes.
				In a LVM-based replication,
				each logical block in a logical volume is mapped to two physical blocks
				on two different physical volumes.
				An Application writing to a logical block now writes to two physical blocks.
				This is known as \emph{LVM Mirroring}.

				It adds some workload to the hosts CPU
				and if the devices are already protected by a RAID
				no addition protection is installed.
				An advantage is the vendor independence
				and as LVMs are usually part of the OS no additional licensing is necessary.
			\item[File-System snapshot] \hfill \\
				This FS snapshot is a prointer based replica,
				that can be impelemented either on the FS or LVM level
				and uses the Copy on First Write (CoFW) principle to create the snapshots.
				It just reqires a fraction of the space used by the production FS.

				When a snaphsot is created,
				a bit and a blockmap are created in the metadata of the Snap FS.
				When a write occurs on the production FS this is held.
				The data previously on this postition is copied to the Snap FS.
				The new data on this postion is then copied to the Snap FS.
				The block and the bit map are changed to represent which data is stored.
				Then the I/Os goes to the production OS.
				%O RLY?
		\end{description}
	% subsubsection host_based_replication (end)

% subsection describe_local_replication_technologies (end)
\subsection{Research factors that determine storage capacity requirements for a save location in pointer-based virtual replication.} % (fold)
\label{sub:research_factors_that_determine_storage_capacity_requirements_for_a_save_location_in_pointer_based_virtual_replication}

% subsection research_factors_that_determine_storage_capacity_requirements_for_a_save_location_in_pointer_based_virtual_replication (end)

\subsection{Research continuous data protection technology and its benefits over array-based replication technologies.} % (fold)
\label{sub:research_continuous_data_protection_technology_and_its_benefits_over_array_based_replication_technologies}

% subsection research_continuous_data_protection_technology_and_its_benefits_over_array_based_replication_technologies (end)

% section local_replication (end)