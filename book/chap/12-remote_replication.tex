\section{Remote Replication} % (fold)
\label{sec:remote_replication}

\subsection{What are the consideration for implementing synchronous remote replication} % (fold)
\label{sub:what_are_the_consideration_for_implementing_synchronous_remote_replication}
	The major advantage of synchronous remote replication is its zero RPO.
	As changes are committed directly to the remote site,
	it can be used as fall back in case of a failure at the production site.
	Though it is important that the distances between production host
	and remote site can not be extended over large distances.
	Larger distances lead to a longer response time for the application,
	which can be a problem depending on the type of application.
	The distance is also determined by the connection between source and target
	and the quality of the connection.
% subsection what_are_the_consideration_for_implementing_synchronous_remote_replication (end)

\subsection{Explain the RPO that can be achieved with remote replication techniques.} % (fold)
\label{sub:explain_the_rpo_that_can_be_achieved_with_remote_replication_techniques}

	\begin{description}
		\item[Synchronous replication] \hfill \\
			Zero RPO, as changes to the data are only acknowledged to
			the application when they were made at the production host
			and on the remote replication site.
		\item[Asynchronous replication] \hfill \\
			Depending on the buffer size and the connection the RPO can be near zero,
			but never zero.
			If the bandwith is a problem,
			the buffer has to be larger enough to be ready for peaks.
		\item[disk-buffered replication] \hfill \\
			The RPO is usually in the order of hours,
			as it takes time to create the local replica
			and then send the data to the remote site.
	\end{description}

% subsection explain_the_rpo_that_can_be_achieved_with_remote_replication_techniques (end)

\subsection{Discuss the effects of a bunker failure in a three site replication.} % (fold)
\label{sub:discuss_the_effects_of_a_bunker_failure}

	\begin{description}
		\item[Multihop - synchronous + disk-buffered] \hfill \\
			No further synchronisation can take place.
			The RPO depends on the last synchronisation between the bunker
			and the remote site.
			It is usually in the order of hours.
		\item[Multihop - synchronous + asynchronous] \hfill \\
			No further synchronisation can take place.
			The RPO depends on the last synchronisation between the bunker
			and the remote site.
			It is usually in the order of minutes.
		\item[Multitarget] \hfill \\
			Synchronisation will continue asynchronously.
			The RPO denpends on the buffer size and connection
			as described in section \ref{sub:explain_the_rpo_that_can_be_achieved_with_remote_replication_techniques}.
	\end{description}

% subsection discuss_the_effects_of_a_bunker_failure_in_a_three_site_replication (end)

\subsection{Discuss the effects of a source failure in a three site replication and available recover options.} % (fold)
\label{sub:discuss_the_effects_of_a_source_failure_in_a_three_site_replication_and_available_recover_options}

	\begin{description}
		\item[Multihop - synchronous + disk-buffered] \hfill \\
		\item[Multihop - synchronous + asynchronous] \hfill \\
		\item[Multitarget] \hfill \\
	\end{description}

% subsection discuss_the_effects_of_a_source_failure_in_a_three_site_replication_and_available_recover_options (end)

% section remote_replication (end)