\section{Fibre Channel Storage Area Networks} % (fold)
\label{sec:fibre_channel_storage_area_networks}

\subsection{What is zoning?} % (fold)
\label{sub:what_is_zoning}
	Zoning is an FC switch function
	that enables node ports within the fabric to be logically segmented
	into groups and to communicate with each other within the group.

	Whenever a change takes place in the name server database,
	the fabric controller sends a Rgeistered State Change Notification(RSCN)
	to all the nodes impacted.
	This can generated large amounts of fabric-management traffic,
	which might impact host-to-storrage traffic.
	With enabled zoning,
	only the affected nodes in a zone get the RSCN from the fabric.

\subsubsection{Where is WWN zoning preferred over port zoning?} % (fold)
\label{ssub:where_is_wwn_zoning_preferred_over_port_zoning}
	A setup where parts are regularily moved
	and new targets are added.
% subsubsection where_is_wwn_zoning_preferred_over_port_zoning (end)

\subsubsection{Where is port zoning preferred over WWN zoning?} % (fold)
\label{ssub:where_is_port_zoning_preferred_over_wwn_zoning}
	In a setup where targets will not be switched
	but be replaced in case of failure.
	Long time Archiving might be a usecase.	
% subsubsection where_is_port_zoning_preferred_over_wwn_zoning (end)
% subsection what_is_zoning (end)

\subsection{Describe the process of assigning an FC address to a node when logging into a network for the first time!} % (fold)
\label{sub:describe_the_process_of_assigning_an_fc_address_to_a_node_when_logging_into_a_network_for_the_first_time}
	This process is called \emph{fabric login}(FLOGI) which is performed between an N and a F\_Port.
	To log on o the fabric the node sends a FLOGI frame with its WWNN and WWPN parameters
	to the login service at the predefined FC address \texttt{FFFFFE} (Fabric Login Server).
	The service accepts the login with an Accept (ACC) frame
	which includes the assigned FC address for the node.
	Now the node registers itself with the local Name Server on the switch,
	with its WWNN, WWPN, port type, class of service and assigned FC address.
	After loging in the N\_Port can query the name server database for information
	about all other logged in ports.
% subsection describe_the_process_of_assigning_an_fc_address_to_a_node_when_logging_into_a_network_for_the_first_time (end)

\subsection{Discuss the roles of the name server and fabric controller in an FC-switched fabric!} % (fold)
\label{sub:discuss_the_roles_of_the_name_server_and_fabric_controller_in_an_fc_switched_fabric}
	The \emph{name server} keeps track of every connected node.
	It enables information exchange between the switches inside the fabric,
	to ensure a distributed and synchronized name service.
	It is located at the predefined adress \texttt{FFFFFE}.

	The \emph{fabric controller},
	which is located at the predefined address \texttt{FFFFFD},
	provides services for node ports and switches.
	It manages the Registerd State Change Notifications (RSCNs)
	who are send to registerd node ports in the fabric
	in case off a change in the fabric.
	It also generate Switch Registerd State Change Notifications (SW-RSCNs)
	to every other domain (swith) in the fabric to
	keep their name services updated.
% subsection discuss_the_roles_of_the_name_server_and_fabric_controller_in_an_fc_switched_fabric (end)

\subsection{How does flow control work in an FC network?} % (fold)
\label{sub:how_does_flow_control_work_in_an_fc_network}

	\subsubsection{BB\_Credit} % (fold)
	\label{ssub:bb_credit}
		This mechanism limits the maximal nuember of frames
		that can be present over a link at any point in time.
		The transmitting port keeps track of the free receiver buffers
		and sends as long as the count is greater than 0.
		Receiver Ready (R\_RDY) primitives are used to indicate a freed buffer.
	% subsubsection bb_credit (end)

	\subsubsection{EE\_Credit} % (fold)
	\label{ssub:ee_credit}
		Initiator and target can use this mechanism to exchange EE\_Credit parameters.
		The function is similar to BB\_Credit,
		but provides only flow controll for traffic in the classes 1 and 2.
	% subsubsection ee_credit (end)
% subsection how_does_flow_control_work_in_an_fc_network (end)

\subsection{Explain storage migration using block-level storage virtualization. Compare this to traditional migration methods.} % (fold)
\label{sub:explain_storage_migration_using_block_level_storage_virtualization_compare_this_to_tranditional_migration_methods}
	The virtualization layer allows a better migration in comparison to tranditional migration methods.
	As this layer handles the access to the underlying block devices,
	migration can take place while they are online.
	Applications can still access the data
	via the addresses on the virtualization layer.
	As the virtualization layer handles the back-end migration
	it is also possible to migrate in between data centers.\\
	These techniques are especially usefull in a multivendor environment.
% subsection explain_storage_migration_using_block_level_storage_virtualization_compare_this_to_traditional_migration_methods (end)

\subsection{How do VSANs improve the manageability of an FC SAN?} % (fold)
\label{sub:how_do_vsans_improve_the_manageability_of_an_fc_san}
	Virtual SANs can be configured more easily as they dont require any recabeling,
	only the VSAN configuration has to be changed.
	As they are also isolated form each other,
	VSANs enhance the security
	and in case of traffic diruption
	these are contained inside the VSAN
	and do not propagate through the entire physical network.
% subsection how_do_vsans_improve_the_manageability_of_an_fc_san (end)

% section fibre_channel_storage_area_networks (end)