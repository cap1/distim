\section{Introduction to Business Continuity} % (fold)
\label{sec:introduction_to_business_continuity}

\subsection{Formulas in Business Continuity} % (fold)
\label{sub:formulas_in_business_continuity}

	\subsubsection{Mean Time Between Failure (MTBF)} % (fold)
	\label{ssub:mean_time_between_failure}
		It is the average time available for system or component
		to perform its normal operations between failures expressed in hours.
	% subsubsection mean_time_between_failure_ (end)

	\subsubsection{Mean Time To Recover (MTTR)} % (fold)
	\label{ssub:mean_time_to_recover}
		Describes the time to required to repair a failed component.
		It includes the time it takes to detect the fault,
		until it is fully resolved
		and the system is running again.
	% subsubsection mean_time_to_recover_ (end)

	\subsubsection{Information Availability} % (fold)
	\label{ssub:availability}
		Information Availability (IA) is the time period during
		which a system is in condition to perform its intendet function.
		It can be described with system up- and downtime \ref{eq:IA_system}
		or with MTBF and MTTR \ref{eq:IA_MTBF_MTTR}.

		\begin{equation}
			IA = system\ uptime/(system uptime + system downtime)
		\end{equation}
		\label{eq:IA_system}

		\begin{equation}
			IA = MTBF/(MTBF + MTTR)
		\end{equation}
		\label{eq:IA_MTBF_MTTR}
	% subsubsection availability (end)
% subsection formulas_in_business_continuity (end)

\subsection{Business Continuity Terms} % (fold)
\label{sub:business_continuity_terms}
	\subsection{Recovery-Point Objective} % (fold)
	\label{sub:recovery_point_objective}
		Point in time to which systems and data must be recovered after an outage.
		This defines the amount of data that can be lost that a business can endure.
		Based on the \emph{RPO} business can schedule backups.
	% subsection recovery_point_objective (end)

	\subsection{Recovery-Time Objective} % (fold)
	\label{sub:recovery_time_objective}
		The time within a system
		and application must be recovered after an outage.
		The \emph{RTP} defines the amount of downtime
		that a business can endure.
		Based on the RTO businesses can plan their disaster recovery plans.
	% subsection recovery_time_objective (end)

	\subsection{Hot- and Coldsite} % (fold)
	\label{sub:hot_and_coldsite}
		A \emph{Hotsite} is a site where a business can move its operations to
		in the event of a disaster.
		The site contains everything necessary,
		including hardware and software.
		The equipement is available and running at all times.

		A \emph{Coldsite} contains the same things as a Hotsite,
		but it just contains the minimum require infrastructure.
		Also it is not active.
	% subsection hot_and_coldsite (end)
% subsection business_continuity_terms (end)

% section introduction_to_business_continuity (end)