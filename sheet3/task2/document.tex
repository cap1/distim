\documentclass{article}
\usepackage[utf8]{inputenc}
\usepackage{graphicx}
\usepackage{amsmath}
\usepackage{booktabs}
\usepackage{textcomp}
\usepackage{multirow}
\usepackage{color}

%use (a) for numbering subsections
\renewcommand*\thesubsection{(\alph{subsection})}
\title{Aufgabenblatt 3}
\author{Christian Müller, Ralph Krimmel \& Sebastian Albert }

\begin{document}

\maketitle

\section*{Assignment 2 - IP SAN and FCoE}

\subsection{How does iSCSI handle the process of authentication?\\Research the available options!}
	iSCSI usually handles the authentication with Challenge-Handshake Authentication Protocol (CHAP).
	This protocol requires a shared secret.
	The initiator sends the random value to the target 
	and both calculate the digest of the shared secret and this random value.
	The target sends the digest back to the initiator who completes authentication,
	if the digest and therby the shared secrets are identical.

\subsection{List some of the data storage applications that could benefit from an IP SAN!}

\subsection{What are the major performance considerations for FCIP?}

\subsection{The IP bandwith provided for FCIP connectivity seems to be constrained.
	Discuss its implications if the SANs that are merged are fairly large,
	with 500 ports on each side, and the SANs at both ends are constantly reconfigured!}

\subsection{Why does the lossy nature of standard Ethernet make it unsuitable for a layered FCoE implementations?
	How does Converged Enhanced Ehternet (CEE) address this problem?}

\end{document}
