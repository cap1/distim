\documentclass{article}
\usepackage[utf8]{inputenc}
\usepackage{graphicx}
\usepackage{amsmath}
\usepackage{booktabs}
\usepackage{textcomp}
\usepackage{multirow}
\usepackage{color}

%use (a) for numbering subsections
\renewcommand*\thesubsection{(\alph{subsection})}

\begin{document}

\title{Aufgabenblatt 2}
\author{Christian Müller, Ralph Krimmel \& Sebastian Albert }
\maketitle

\section*{Assignment 1 - RAID}

\subsection{Why is RAID 1 not a substitute for a backup?}
	In a RAID 1 setup all the data written is mirrored to at least a second drive.
	That means that all written should be dupilcated,
	so in the case of a single disk failure the impact on recovery is the least among all RAID implementations.
	Althoug there are some cases were problems can occur.

	The Disk can have a seperate write cache independent of the RAID controller.
	If a loss of power occurs and different amounts of data have been written to the disks,
	consistency problems can occur.

	A backup is usually considered to be a copy of data from a distinct point in time.
	In contrast,
	RAID 1 just keeps a duplicate of the current data 
	and it is impossible to revert to an early state of the data.
	Furthermore a backup is usually not stored in the same disk array	as the live system
	due to be safe from failures in higher systems like the disk array or the RAID controller itself.
	Good backup stragtegies also keep the backupped data in a different data center,
	to be protected against severe problems like thunderstorms or fires.

	All the things are not part of RAID 1,
	so it should only be considered to enhance availabilty
	and not as a subsitutie for backups.

\subsection{Why is RAID 0 not an option for data protection and high availability?}

\subsection{Explain the process of data recovery in case of a drive failure in RAID 5}

\subsection{What are the benefits of using RAID 3 in a backup application?}

\subsection{Discuss the impact of random and sequential I/O in different RAID configurations!}
\subsubsection*{RAID 3}

\subsubsection*{RAID 5}

\subsubsection*{RAID 1/0}

\end{document}
