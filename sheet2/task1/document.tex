\documentclass{article}
\usepackage[utf8]{inputenc}
\usepackage{graphicx}
\usepackage{amsmath}
\usepackage{booktabs}
\usepackage{textcomp}
\usepackage{multirow}
\usepackage{color}

%use (a) for numbering subsections
\renewcommand*\thesubsection{(\alph{subsection})}
\title{Aufgabenblatt 2}
\author{Christian Müller, Ralph Krimmel \& Sebastian Albert }

\begin{document}

\maketitle

\section*{Assignment 1 - RAID}

\subsection{Why is RAID 1 not a substitute for a backup?}
	In a RAID 1 setup all the data written is mirrored to at least a second drive.
	That means that all written should be dupilcated,
	so in the case of a single disk failure the impact on recovery is the least among all RAID implementations.
	Althoug there are some cases were problems can occur.

	The Disk can have a seperate write cache independent of the RAID controller.
	If a loss of power occurs and different amounts of data have been written to the disks,
	consistency problems can occur.

	A backup is usually considered to be a copy of data from a distinct point in time.
	In contrast,
	RAID 1 just keeps a duplicate of the current data 
	and it is impossible to revert to an early state of the data.
	Furthermore a backup is usually not stored in the same disk array	as the live system
	due to be safe from failures in higher systems like the disk array or the RAID controller itself.
	Good backup stragtegies also keep the backupped data in a different data center,
	to be protected against severe problems like thunderstorms or fires.

	All the things are not part of RAID 1,
	so it should only be considered to enhance availabilty
	and not as a subsitutie for backups.

\subsection{Why is RAID 0 not an option for data protection and high availability?}
	As RAID 0 makes use of \emph{Striping},
	which spreads the data to at least two disks,
	while not keeping any parity information.
	It was designed with high I/O throughput
	and high numbers of simoultaneous read and write operations in mind.

	As no parity information is stored,
	the failure of a single drive leads to the loss of all data.
	This quite obviously contradics data protection and high availability.

\subsection{Explain the process of data recovery in case of a drive failure in RAID 5}
	RAID 5 uses striping of data and distributes the parity information over all disks.
	In the case of a failure of a single  data can be restored by reversing the XOR operation
	that was used to generate the parity information.
	As the partiy information is distributed over all drives in RAID 5,
	some of it is lost if a disk fails.
	This is not a problem becaus it can be generated again form the data itself.
	The recovery works blockwise,
	like the data is stripe accross the drives.
	The remaining data and if availabel the parity information get XORed with each other
	and thereby generates the missing block.
	As described before,
	if the parity information is missing it is can be calculated by XORing the data blocks 
	with themselfs.

	In an ideal case,
	this data is then saved to an hot spare disk,
	which repalacs the disk that failed.
	%TODO this probably needs a figure

\subsection{What are the benefits of using RAID 3 in a backup application?}
	With RAID 3 one benefits from havin fast sequential write and reads like in RAID 0.
	RAID 3 adds parity information to the setup of RAID 1 on a seperate disk.
	This allows to recover from the failure of a one disk in the setup.
	Also RAID 3 does not need huge amounts of additional diskspace,
	as the parity information is the only data generated additionaly.

\subsection{Discuss the impact of random and sequential I/O in different RAID configurations!}
\subsubsection*{RAID 3}
	\emph{Sequential I/O} can be handle very well by RAID 3,
	as it can spread the I/O with striping over all the disks.
	\emph{Random I/O} reads can be handle very well too,
	but it is not that good with sequential writes.
	This is due to the write penalty,
	which occurs in the creation of the parity information.
	Overall the speed can be described by $(n-1)X$,
	where n is the number of used drives and X the speed of the drives.

\subsubsection*{RAID 5}
	\emph{Sequential I/O} can be handle very well by RAID 5,
	as it can be spread over all the disk by striping.
	\emph{Random I/O} can be handle well to.
	The write penalty is the same as in RAID 3,
	but as the parity information are not stored on a dedicated disk,
	it can also be spread.
	The formula described in RAID 3 also holds for RAID 5.

\subsubsection*{RAID 1/0}
	RAID 1/0 delivers the best performance for \emph{Sequential I/O} and 
	\emph{Random I/O}, as striping can be used without the write penalty
	for the parity information.
	The perfomance increases linear with the number of disks.

\end{document}
