\documentclass{article}
\usepackage[utf8]{inputenc}
\usepackage{graphicx}
\usepackage{amsmath}
\usepackage{booktabs}
\usepackage{textcomp}
\usepackage{multirow}
\usepackage{color}
\usepackage{listings}

\lstset{
	numbers=left,
	basicstyle=\small,
	tabsize=3,
	showspaces=false,
	showtabs=false,
	showstringspaces=false
}

\begin{document}
\title{Assignment 2.3}
\author{Christian Müller, Ralph Krimmel \& Sebastian Albert}
\maketitle
\section*{(a)}
\begin{itemize}
\item 5 disks available with 931 GB in total, no buying option, 730 GB of data
\begin{itemize}
\item maximum 20\% redundancy possible (745 GB remaining then)
\item RAID 1, 4, 6 disqualify
\end{itemize}
\item 15\% random writes, 85\% random reads
\begin{itemize}
\item among the remaining options RAID 3 and 5, RAID 5 performs better
\end{itemize}
\end{itemize}
\section*{(b)}
\begin{itemize}
\item 900 GB of data, increasing by up to 30\% (270 GB), total necessity 1170 GB
\begin{itemize}
\item 7 disks needed non-redundantly if each has a capacity of 186 GB
\item new disks can be bought (actually have to) as only 6 are present
\end{itemize}
\item 40\% writes, 60\% reads
\begin{itemize}
\item high write ratio induces much write penalty with parity checks
\item recommendation therefore: RAID 1+0
\item 7 mirror sets needed, so 14 disks needed
\item 6 disks are present: 8 need to be bought (cost)
\end{itemize}
\end{itemize}
\end{document}
